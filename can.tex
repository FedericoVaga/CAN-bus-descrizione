\documentclass[a4paper,10pt]{book}
\usepackage[utf8x]{inputenc}
\usepackage{graphicx}
\DeclareGraphicsRule{.eps}{eps}{}{}
\usepackage{hyperref}

\title{Controller Area Network}
\author{Federico Vaga}
\date{Giugno 2012}

\begin{document}
% stampa la licenza CC BY-NC-SA_2.5_IT
\input{licenza/CreativeCommons/BY-NC-SA_2.5_IT/BY-NC-SA_2.5_IT.tex}

\tableofcontents
\listoffigures

\chapter{Introduzione}
\section{Storia del protocollo}
Il \textit{Controller Area Network}, abbreviato in \textit{CAN}, è un
protocollo di comunicazione seriale ideato da Bosch negli anni '80.
Questo protocollo è stato espressamente progettato per resistere a
forti interferenze dovute ad onde elettromagnetiche esterne; questo
suo punto di forza fa si che il protocollo CAN trovi largo utilizzo
nel settore automotive e nel settore dell'automazione industriale.
Il lavoro a questo protocollo iniziò nel 1983 e fu ufficialmente
rilasciato nel 1986; nel 1987 fu commercializzato il primo controller
CAN prodotto da Intel e Philips; nel 1991 Bosch rilascia le
specifiche per il CAN 2.0.
\newline

Dal 1996 negli Stati Uniti d'America tutti i veicoli commercializzati
devono sottostare allo standard \textit{OBD-II} per la diagnostica di
bordo, e il protocollo CAN è uno dei cinque protocolli consentiti. In
Europa dal 2001 è obbligo rispettare lo standard EOBD per i veicoli a
benzina, mentre dal 2004 per quelli a diesel; anche per lo standard
europeo il protocollo CAN è uno di quelli ammessi.
\newline

In protocollo CAN è riconosciuto come standard con il rilascio dell'
ISO 11898.

\section{Caratteristiche principali}
Come accennato in precedenza, il protocollo CAN è stato progettato
per essere molto robusto alle interferenze ed essere immune da errori
di trasmissione. Questa grande resistenza è dovuta all'applicazione
della specifica RS-485 del livello fisico del modello OSI. La
specifica definisce la trasmissione di un dato mediante la differenza
di potenziale su due differenti fili; quindi avremo uno stato logico
alto quando la differenza di potenziale è superiore a $0.2V$, basso
quando inferiore. La resistenza al rumore può essere aumentata
realizzando un doppino intrecciato (twisted pair).
\newline

Il protocollo garantisce una velocità di trasmissione massima di
$1\frac{Mb}{s}$ per distanze non superiori a $40m$; distanze
maggiori sono raggiungibili a velocità inferiori.
\newline

I protocollo è basato su una comunicazione broadcast, quindi tutti i
nodi ascoltano le comunicazioni che avvengono sul bus anche quando
non sono dirette a loro. In questo modo è molto semplice rimuovere o
aggiungere nodi alla rete senza dover effettuare alcuna modifica
hardware o software. La trasmissione è message-ortiended, ovvero
basata sui messaggi; ogni messaggio contiene un identificativo
univoco che identifica l'origine del messaggio all'interno della rete

\chapter{La specifica}
La standard CAN descrive i livelli fisico e data link del modello
\textit{ISO/OSI} per la connessione fra calcolatori.
\section{Livello fisico}
Il livello fisico è costituito da due cavi denominati \textit{CAN\_H}
e \textit{CAN\_L}. I livelli logici trasmessi sono definiti
\textit{dominante} e \textit{recessivo}; non c'è una corrispondenza
diretta con i classici stati logici 1 e 0, perché questo dipende da
come i nodi si interfacciano con il bus. Durante la trasmissione i
bit dominanti hanno sempre la precedenza sui bit recessivi; se
durante una trasmissione i nodi inviano un bit recessivo ma un
diverso nodo invia un bit dominante, allora il bit che effettivamente
è trasmesso sul bus è quello dominante. Tutti i nodi che
trasmettevano un bit recessivo si accorgono della presenza del bit
dominante, quindi smettono di trasmettere e rimangono in ascolto.

I nodi possono essere \textit{OR-Wired} oppure \textit{AND-Wired};
nel primo caso avremo la corrispondenza dominante:1, recessivo:0; nel
secondo caso la corrispondenza è dominante:0 recessivo:1

%FIXME figura dei circuit per fare or wired e and wired

Questo livello è diviso in tre sottolivelli: PLS, PMA, MDI
\subsection{Sottolivello PLS}
La sigla PLS corrisponde a \textit{Physical Signaling}. Questo
sottolivello si occupa della sincronizzazione e della temporizzazione
dei segnali attraverso la tecnica di \textit{bit stuffing}, ovvero
l'inserimento di un bit di controllo all'interno della comunicazione.
Nel nostro caso, ogni 5 bit viene inserito un bit di polarità
inversa; questo facilita la sincronizzazione fra i nodi.
\newline

Il \textit{bit-time}, ovvero la durata di un bit, è suddivisa in
quattro parti dette \textit{segment}, ciascuna multipla del
\textit{time-quantum} \textit{TQ}. Il TQ è l'unità di tempo usata dal
protocollo per suddividere il tempo; esso è definito mediante l'uso di
un oscillatore locale. Le quattro parti che compongono la trasmissione
di un bit sono:
\begin{description}
 \item[Syncronization] dura 1TQ ed è utilizzata per inserire fronti
di commutazione e sincronizzazione fra i nodi.
 \item[Propagation time] può durare da 1 a 8 TQ, è utilizzato per
compensare i ritardi di trasmissione/ricezione.
 \item[Phase 1 e 2] entrambe i segmenti hanno la stessa durata che
può variare da 1 a 8 TQ. Fra i due segmenti viene posto il
\textit{sample-point}, ovvero il punto in cui viene letto il livello
presente sul bus. L'ampiezza di questi due segmenti è regolata di modo
da compensare le fasi. Il sistema verifica la sincronizzazione fra
nodi durante la transizione fra recessivo e dominante e calcola la
distanza del fronte con il sample-point; questa distanza
\underline{deve} essere sempre la stessa. I fronti sono rilevati ad
ogni TQ e viene confrontato lo stato della linea con lo stato
dell'ultimo sample-point. Per mantenere costante la distanza fra il
fronte e il sample-point, i segmenti di fase devono essere
opportunamente dimensionati; è possibile definire un limite
all'allungamento o accorciamento di queste fasi medianto il parametro
di \textit{Syncronization Jump Width} \textit{SJW} che può variare da
1 a 4 TQ; questo procedimento è detto \textit{bit resynchronization}.
Un metodo alternativo è quello di resettare il bit time ogni volta
che si presenza un fronte, tecnica denominata \textit{hard
synchronization}
\end{description}

%FIXME immagine per spiegare meglio

\subsection{Sottolivello PMA}
La sigla PMA corrisponde a \textit{Physical Medium Attachment}. Lo
scopo di questo livello è definire come i nodi vengono collegati al
bus. Il PMA prevede 3 componenti:
\begin{description}
 \item[Transceiver] è un buffer bidirezionale per l'accesso al bus
quindi rileva la differenza fra CAN\_H e CAN\_L
 \item[Controller] è il componente che si occupa della trasmissione e
ricezione dei dati sul bus
 \item[Microcontroller] è il dispositivo che si occupa di gestire
transceiver e controller, quindi di decidere cosa trasmettere e
analizzare i dati ricevuti.
\end{description}

%FIXME immagine

\subsection{Sottolivello MDI}
La sigla MDI corrisponde a \textit{Medium Dependent Interface}. A
questo livello sono definite le caratteristiche elettriche e fisiche:
come sono connessi i nodi, impedenze di adattamento, ritardi di linea
ecc.

\section{Livello Data-link}
Il livello data-link si occupa della definizione di come sono
strutturati i messaggi, di come vengono trattati gli errori, di come
vengono convalidati i messaggi, di come isolare i nodi malfunzionanti
e delle regole di arbitraggio.
\subsection{Formato messaggi}
Il protocollo CAN riconosce 5 tipi di formati messaggi:
\begin{description}
 \item[Data Frame] è un messaggio che contiene dati utili; ogni
ricevente può decidere se usarli o scartarli.
 \item[Remote Frame] è un messaggio simile al data frame, ma privo
della parte dati; esso è utilizzato come \textit{data request}.
 \item[Error Frame] è un messaggio inviato da un nodo che rileva un
errore e provoca il re-invio di un messaggio. Se un nodo rileva un
alto tasso di errore si esclude dalla rete.
 \item[Overload] è un messaggio inviato da un nodo occupato per
chiedere di ritardare la spedizione del prossimo messaggio
 \item[Interframe Space] è un messaggio che precede ciascun data
frame e remote frame di modo da permettere la separazione fra questi
due tipi di messaggio.
\end{description}

\subsubsection{Data Frame e Remote Frame}
Un \textit{Data Frame} o \textit{Remote Frame} è costituito da 10
campi:

\begin{description}
 \item[SOF]: Start Of Frame, 1 bit dominante per per indicare
l'inizio del frame
 \item[Arbitration field]: può essere lungo 11 o 29 bit; esso è un
identificatore di messaggio
 \item[RTR]: Remote Transmission Request, 1bit che serve a distinguere
fra Data Frame e Remote Frame; è dominante nel caso del Data Frame,
mentre è recessivo nel caso del Remote Frame. In questo modo la
trasmissione dati ha priorità maggiore della richiesta.
 \item[IDE]: 1 bit riservato
 \item[r0]: 1 bit riservato
 \item[DLC]: 4 bit per indicare la lunghezza del campo Data Field
 \item[Data Field]: da 0 a 8 Byte di dati. Nel caso del Remote Frame,
questo campo è assente, quindi DLC è 0.
 \item[CRC]: 15bit di CRC più uno aggiuntivo sempre recessivo.
 \item[ACK]: 2 bit recessivi, denominati \textit{Slot} e
\textit{Delimiter}; se almeno un nodo destinatario ha ricevuto il
messaggio \textit{Slot} viene da esso sovrascritto con un bit
dominante. Questo garantisce che il messaggio è stato consegnato,
anche se non sappiamo chi ha ricevuto il messaggio.
 \item[EOF]: End Of Frame, 7 bit recessivi indicano la fine del frame.
\end{description}

\subsubsection{Error Frame}
\subsubsection{Overload}
\subsubsection{Interframe Space}
Come accennato è frame usato per separare i Data Frame dai Remote
Frame. Questo frame è costituito da una sequenza di bit tutti
recessivi. Durante la trasmissione dei primi tre bit recessivi, detti
\textit{intermission}, nessun nodo può inviare Data Frame o Remote
Frame; tuttavia possono essere inviati frame di altro tipo. A seguire
una sequenza di 8 bit recessivi utilizzati da un nodo soggetto ad
errori come attesa per inviare nuovamente il proprio messaggi; altri
nodi però possono iniziare una nuova trasmissione, quindi il nodo
soggetto ad errore può decidere se accettare il nuovo frame in
trasmissione o rifiutarlo a secondo che il suo stato sia,
rispettivamente, \textit{error passive} o \textit{non error passive}.
Infine una sequenza di lunghezza non specificata di bit recessivi per
indicare lo stato di \textit{Bus Idle}, ovvero non ci sono
trasmissioni in corso.


\end{document}
